\documentclass[10pt, onecolumn]{article}
\usepackage{amsmath}
\usepackage{enumerate}
\usepackage{enumitem}
\usepackage{listings}
\usepackage[utf8]{inputenc}
\usepackage{amssymb}
\usepackage{tabularx}

\usepackage{booktabs}
\usepackage{multirow}
\usepackage{siunitx}
\lstset{
    frame=single,
    breaklines=true
}
\usepackage{hyperref}
%\usepackage{atbegshi}
%\AtBeginDocument{\AtBeginShipoutNext{\AtBeginShipoutDiscard}}
%\newcommand{\solution}{\noindent \textbf{Solution:}}
%\documentclass[twoside]{article}
\usepackage[a4paper,outer=1.5cm,inner=1.5cm,top=1.75cm,bottom=1.5cm]{geometry}
\def\mytitle{\textbf{JETSON-PI}}
\def\myauthor{DUDEKULA USENI}
\def\contact{r171099@rguktrkv.ac.in}
\def\myauthor{DUDEKULA USENI - AHILAN R}
\def\contact{r171099@rguktrkv.ac.in - ra1atcuk@gmail.com}
\def\mymodule{Future Wireless Communication (FWC)}

%\thiswatermark{\centering \put(-15,-100.0){\includegraphics[scale=0.4]{iith.png}} }
\title{\mytitle}
\author{\myauthor\hspace{1em}\\\contact\\FWC22098-FWC22090 -\hspace{0.5em}IITH\hspace{0.5em}\mymodule\hspace{6em}}

\begin{document}
\maketitle
\begin{enumerate}
\section{Abstract}
\item[\textbf{}] Time series autocorrelation

We would like to experiment and see 
Can raspberry pi and/or nano do time series autocorrelation for large data sets. Data set usually will contain about 40e6 to 1.5 e8 points and will be sampled uniformly

\section{Perfomance Comparision Tables Of PI And JETSON}

\begin{tabularx}{0.9\textwidth} { 
  | >{\raggedright\arraybackslash}X 
  | >{\centering\arraybackslash}X 
  | >{\centering\arraybackslash}X
  | >{\raggedleft\arraybackslash}X | }
\hline
\multicolumn{3}{|c|}{\textbf{METHOD-1 PYTHON IMPLEMENTATION}} \\
\hline
\textbf{SAMPLES} & \textbf{PI} & \textbf{JETSON}\\
\hline
50K &   36  minutes  & 9 minutes \\ 
\hline
100K &  2.69 hrs     & 52.67 minutes \\  
\hline
200K &  limit time exceed         & limit time exceed \\  
\hline
\end{tabularx}\\


\begin{tabularx}{0.9\textwidth} { 
  | >{\raggedright\arraybackslash}X 
  | >{\centering\arraybackslash}X 
  | >{\centering\arraybackslash}X
  | >{\raggedleft\arraybackslash}X | }
\hline
\multicolumn{3}{|c|}{\textbf{METHOD-2 STATSMODEL}} \\
\hline
\textbf{SAMPLES} & \textbf{PI} & \textbf{JETSON}\\
\hline
50K  &   1.58 sec  & 9.69 sec \\ 
\hline
100K &  2.06 sec   & 28.90 sec \\  
\hline
200K &  2.29 sec   & 1.88 minutes \\  
\hline
%500K &  5    &    13.91 minutes      \\
%\hline
1M  &  3.57 sec &  51.58 minutes            \\     
\hline
10M &  26.03 sec &  limit time exceed           \\
\hline
20M &   Killed  &  killed      \\
\hline
\end{tabularx}\\


\begin{tabularx}{0.9\textwidth} { 
  | >{\raggedright\arraybackslash}X 
  | >{\centering\arraybackslash}X 
  | >{\centering\arraybackslash}X
  | >{\raggedleft\arraybackslash}X | }
\hline
\multicolumn{3}{|c|}{\textbf{METHOD-3 NUMPY.CORRELATE}} \\
\hline
\textbf{SAMPLES} & \textbf{PI} & \textbf{JETSON}\\
\hline
50K  &   4.701 sec  & 6.80 sec \\ 
\hline
100K &  11.37 sec   & 27.18 sec \\  
\hline
200K &  1.89 minutes   & 1.89 minutes \\  
\hline
%500K &      &  13.96 minutes       \\
%\hline
1M  &  1.90 minutes & 52.96 minutes             \\     
\hline
10M &  limit time exceed & limit time exceed             \\
\hline
20M &   Killed  &  killed      \\
\hline
\end{tabularx}\\


\begin{tabularx}{0.9\textwidth} { 
  | >{\raggedright\arraybackslash}X 
  | >{\centering\arraybackslash}X 
  | >{\centering\arraybackslash}X
  | >{\raggedleft\arraybackslash}X | }
\hline
\multicolumn{3}{|c|}{\textbf{METHOD-4 FOURIER TRANSFORM}} \\
\hline
\textbf{SAMPLES} & \textbf{PI} & \textbf{JETSON}\\
\hline
50K  &   0.08 sec  & 1.49 sec \\ 
\hline
	100K &  0.16 sec   & 2.70 sec \\  
\hline
	200K &  0.39 sec   & 5.27 sec \\  
\hline
%	500K &      &      12.49 sec  \\
%\hline
	1M  &  0.51 sec &    24.98 sec          \\     
\hline
	10M &  44.20 sec &    52.19 sec         \\
\hline
	20M &   Killed  &  killed      \\
\hline
\end{tabularx}\\
\section{conclusion}
Overall, RaspberryPi is faster to execute the autocorrelation than Jetson Nano. However, both have limitaions depending on the number of samples taken and the method of implementation. For example, when 20 million samples were taken both output produced "killed".
\section{Software}
 Download the following code
 \begin{lstlisting}
https://github.com/dudekulauseni123/FWC0982022
 \end{lstlisting}

\end{enumerate}
\end{document}
%\end{document}